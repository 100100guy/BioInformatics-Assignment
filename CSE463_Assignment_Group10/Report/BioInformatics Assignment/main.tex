\documentclass{report}
\usepackage[english]{babel}
\usepackage{placeins}
\usepackage[utf8]{inputenc}
\usepackage{multicol}
\usepackage{multirow}
\usepackage{url}
\usepackage{graphicx}
\usepackage{amsfonts}
\usepackage[tbtags]{amsmath}
\usepackage{amsmath}
\usepackage{caption}
\usepackage{subcaption}
\usepackage{float}
\usepackage{hyperref}
\usepackage[letterpaper,top=2cm,bottom=2cm,left=3cm,right=3cm,marginparwidth=1.75cm]{geometry}
\usepackage{amsmath}
\usepackage{graphicx}
\usepackage{fancyvrb}
\usepackage{mdframed}
\usepackage{float}
\usepackage[colorlinks=true, allcolors=blue]{hyperref}
\usepackage[ruled,linesnumbered]{algorithm2e}
\usepackage{natbib}
\usepackage{booktabs}

\definecolor{lightgray}{RGB}{211, 211, 211}

\title{
	\endgraf\bigskip
	
	\begin{center}
		\Huge {{Report on BioInformatics Assignment} }\\\\
		\vspace{0.5cm}
		\Large {CSE463}
	\end{center}
	\bigskip
	\bigskip
}

\author{
        \large{Mashroor Hasan Bhuiyan(1905069)}\\
        \large{Md. Nafiu Rahman (1905077)}\\
        \large{Kazi Reyazul Hasan(1905082)}\\
	\large{Wasif Jalal (1905084)}\\
        \large{Mubasshira Musarrat(1905088)}\\\\
	\Large{Department of Computer Science and Engineering}\\
        \Large{Bangladesh University of Engineering and Technology (BUET)}
}

\date{
	\endgraf\bigskip
	\Large{\today}
}

\linespread{1.5}
\begin{document}
\maketitle
\tableofcontents

% Report starts


\chapter{Data}
In order to evaluate the performance of our selected methods, modifications, and third-party tools, they were applied to three distinct datasets derived from the TRANSFAC database \citep{Matys2003}, as done in the provided publication \citep{Karaboga2018}.

\section{Datasets Overview}
The first dataset, labeled as \textbf{hm03r}, consists of 10 sequences from humans, each 1500 nucleotides in length. The complexity and diversity of the human genome necessitate sophisticated approaches to accurately identify motifs. The second and third datasets, \textbf{yst04r} and \textbf{yst08r}, are derived from yeast, containing 7 and 11 sequences respectively, each 1000 nucleotides long. The compact nature of yeast genomes presents a different set of challenges, including dense packing of regulatory elements.


\section{Biomarker/Ground Truth}
Exact ground truth (motifs) for selected datasets cannot be found; however, it is possible to examine identified motifs and compare them with known motifs. For this purpose, the \href{https://meme-suite.org/meme/tools/tomtom}{Tomtom} motif comparison tool of the \href{https://meme-suite.org/meme/}{MEME Suite} was used \citep{gupta2007quantifying}.

    
\chapter{Methods}
    \section{Gibbs Sampler}
    Gibbs Sampler is a widely used algorithm for motif discovery in biological sequences. In our study, we applied the Gibbs Sampler algorithm to identify conserved motifs within a set of DNA sequences. The algorithm works by iteratively sampling potential motif occurrences from the input sequences and updating the motif model based on the sampled occurrences. Here's an overview of the steps involved: \newline
    
    \begin{algorithm}[H]
        \SetAlgoLined
        \KwIn{Sequences $S$, Motif length $k$, Number of sequences $t$, Number of iterations $N$}
        \KwOut{Best motifs $M$, Best score $score$, Consensus sequence $consensus$}
        Initialize $M$ by randomly selecting k-mers from $S$\;
        $best\_motifs \gets M$\;
        $best\_score, consensus \gets$ Score($best\_motifs$, $k$)\;
        \For{$j \gets 1$ \KwTo $N$}{
            Choose a random sequence index $i$ from $1$ to $t$\;
            Create profile matrix $P$ excluding motif $M[i]$\;
            Sample a new motif for sequence $S[i]$ using profile-weighted random selection\;
            Calculate score $current\_score$ and consensus for the updated motifs\;
            \If{$current\_score < best\_score$}{
                $best\_motifs \gets M$\;
                $best\_score \gets current\_score$\;
            }
        }
        \Return{$best\_motifs$, $best\_score$, $consensus$}\;
        \caption{Gibbs Sampler Algorithm}
        \end{algorithm}

    
    We implemented the Gibbs Sampler algorithm using Python and tuned its parameters to achieve optimal motif discovery performance.

    \section{Randomized Motif Search}
    Randomized Motif Search is another popular algorithm for motif discovery, particularly useful for its simplicity and effectiveness. In our study, we employed the Randomized Motif Search algorithm to identify conserved motifs across a set of DNA sequences. The algorithm involves the following steps: \newline
    
    \begin{algorithm}[H]
    \SetAlgoLined
    \KwIn{Sequences $S$, Motif length $k$, Number of sequences $t$, Number of iterations $N$}
    \KwOut{Best motifs $M$, Best score $score$, Consensus sequence $consensus$}
    Initialize $M$ by randomly selecting k-mers from $S$\;
    $best\_motifs \gets M$\;
    $best\_score, consensus \gets$ Score($best\_motifs$, $k$)\;
    \While{True}{
        Create profile matrix $P$ based on $M$\;
        Update $M$ by choosing the highest probability k-mer from each sequence using $P$\;
        Calculate score $current\_score$ for the updated motifs\;
        \If{$current\_score < best\_score$}{
            $best\_motifs \gets M$\;
            $best\_score \gets current\_score$\;
        }
        \Else{
            Break\;
        }
    }
    \Return{$best\_motifs$, $best\_score$, $consensus$}\;
    \caption{Randomized Motif Search Algorithm}
    \end{algorithm}

    
    We implemented the Randomized Motif Search algorithm and adjusted its parameters to optimize motif discovery performance for our specific dataset.

\newpage

\section{Enhanced Randomized Motif Search}
We build upon the traditional Randomized Motif Search approach by incorporating multiple iterations of the entire search process to more effectively navigate the solution space. This enhancement aims to mitigate the risk of converging on local optima by initiating the search from various random starting points, thereby increasing the probability of identifying the globally optimal motif set.  Herein, we mention the steps of this augmented algorithm:

\begin{algorithm}[H]
    \SetAlgoLined
    \KwIn{Sequences $S$, Motif length $k$, Number of sequences $t$, Total iterations $T$}
    \KwOut{Best motifs $M$, Best score $score$, Consensus sequence $consensus$}
    Initialize $global\_best\_motifs$ with an arbitrary high score\;
    Initialize $global\_best\_score$ to infinity\;
    \For{$iteration \gets 1$ \KwTo $T$}{
        Seed the random number generator with the current time\;
        Randomly select initial k-mers from $S$ to form $M$\;
        $best\_motifs \gets M$\;
        $best\_score, consensus \gets$ Score($best\_motifs$, $k$)\;
        \While{True}{
            Create profile matrix $P$ based on $M$\;
            Update $M$ by choosing the highest probability k-mer from each sequence using $P$\;
            Calculate score $current\_score$ for the updated motifs\;
            \If{$current\_score < best\_score$}{
                $best\_motifs \gets M$\;
                $best\_score \gets current\_score$\;
            }
            \Else{
                Break\;
            }
        }
        \If{$best\_score < global\_best\_score$}{
            $global\_best\_motifs \gets best\_motifs$\;
            $global\_best\_score \gets best\_score$\;
        }
    }
    \Return{$global\_best\_motifs$, $global\_best\_score$, Consensus($global\_best\_motifs$)}\;
    \caption{Enhanced Randomized Motif Search Algorithm}
\end{algorithm}

 The dynamic seeding mechanism further ensures a varied and comprehensive exploration in each run, enhancing the overall efficiency of the motif finding process.
\newpage

\section{Modified Gibbs Sampler with Exempted Sequences}
The Modified Gibbs Sampler is an extension of the traditional Gibbs Sampler algorithm. This variant introduces the concept of exempted sequences, allowing a predefined number of sequences to remain unaltered during the motif sampling process. This modification aims to explore the impact of stabilizing certain sequences on the overall motif discovery process. Below we detail the steps of the modified algorithm:

\begin{algorithm}[H]
    \SetAlgoLined
    \KwIn{Sequences $S$, Motif length $k$, Number of sequences $t$, Number of iterations $N$, Number of exempt sequences $num\_exempt$}
    \KwOut{Best motifs $M$, Best score $score$, Consensus sequence $consensus$}
    Initialize $M$ by randomly selecting k-mers from $S$\;
    $best\_motifs \gets M$\;
    $best\_score, consensus \gets$ Score($best\_motifs$, $k$)\;
    \For{$j \gets 1$ \KwTo $N$}{
        $exempt\_indices \gets$ Randomly select $num\_exempt$ indices from $1$ to $t$\;
        \For{$i \gets 1$ \KwTo $t$}{
            \If{$i$ not in $exempt\_indices$}{
                Create profile matrix $P$ excluding motifs in $exempt\_indices$\;
                Sample a new motif for sequence $S[i]$ using profile-weighted random selection based on $P$\;
                Update motif $M[i]$ with the sampled motif\;
            }
        }
        Calculate score $current\_score$ and consensus for $M$ including all motifs\;
        \If{$current\_score < best\_score$}{
            $best\_motifs \gets M$\;
            $best\_score \gets current\_score$\;
        }
    }
    \Return{$best\_motifs$, $best\_score$, $consensus$}\;
\caption{Modified Gibbs Sampler Algorithm with Exempted Sequences}
\end{algorithm}

We iteratively sample potential motif occurrences from a set of DNA sequences while fixing a subset of the motifs, chosen randomly at the start of each iteration, to explore the motif space more diversely. By exempting a fraction of sequences from updates, we aim to identify more accurate motifs in selected k-mers.
\newpage

\section{Targeted Gibbs Sampler with Score-Based Selection}
The Targeted Gibbs Sampler is an innovative adaptation of the traditional Gibbs Sampler algorithm used for motif discovery in biological sequences. Unlike the standard approach, which randomly selects sequences for updating, this version prioritizes the update of sequences that, when temporarily excluded, most decrease the overall motif set score. Below, we outline the steps of this targeted algorithm:

\begin{algorithm}[H]
    \SetAlgoLined
    \KwIn{Sequences $S$, Motif length $k$, Number of sequences $t$, Number of iterations $N$}
    \KwOut{Best motifs $M$, Best score $score$, Consensus sequence $consensus$}
    Initialize $M$ by randomly selecting k-mers from $S$\;
    $best\_motifs \gets M$\;
    $best\_score, consensus \gets$ Score($best\_motifs$, $k$)\;
    \For{$j \gets 1$ \KwTo $N$}{
        Identify the sequence $M[i]$ whose removal most decreases the score\;
        Create profile matrix $P$ excluding motif $M[i]$\;
        Sample a new motif for sequence $S[i]$ using profile-weighted random selection based on $P$\;
        Update motif $M[i]$ with the sampled motif\;
        Recalculate score $current\_score$ and consensus for the updated motifs\;
        \If{$current\_score < best\_score$}{
            $best\_motifs \gets M$\;
            $best\_score \gets current\_score$\;
        }
    }
    \Return{$best\_motifs$, $best\_score$, $consensus$}\;
    \caption{Targeted Gibbs Sampler Algorithm with Score-Based Selection}
\end{algorithm}

This modified algorithm introduces a more deliberate approach to motif refinement by focusing on the weakest links within the motif set.


\chapter{Software}
We used MEME and STREME software tools from \href{https://meme-suite.org/meme/tools/streme}{https://meme-suite.org/meme/tools/streme} and \href{https://meme-suite.org/meme/tools/meme}{https://meme-suite.org/meme/tools/meme}. MEME discovers novel, ungapped motifs (recurring, fixed-length patterns) in sequences (sample output from sequences). MEME splits variable-length patterns into two or more separate motifs. STREME discovers ungapped motifs (recurring, fixed-length patterns) that are enriched in your sequences or relatively enriched in them compared to control sequences (sample output from sequences).
    \section{Commands to run}
    To run MEME type the following command - \newline
    \texttt{meme <input file in fasta format> -dna -oc . -nostatus -time 14400 -mod zoops \\-nmotifs 10 -minw <min k> -maxw <max k> -objfun classic -revcomp -markov\_order 0}

    To run STREME type the following command - \newline
    \texttt{streme --verbosity 1 --oc . --dna --totallength 4000000 --time 14400 \\--minw <min k> --maxw <max k> --thresh 0.05 --align center  --p <input file in fasta \\format>} 
    
    \section{Scripts to run}
    To run the tools run tools.sh . The shell script file is-
    \begin{verbatim}
export PATH=\$HOME/meme/bin:\$HOME/meme/libexec/meme-:\$PATH
methods=("meme" "streme")
datasets=("hm03" "yst04r" "yst08r")

for d in \${datasets[@]}; do
    for m in \${methods[@]}; do
        for k in {8..15}; do
            mkdir -p "tool_output/\$d/\$m/k_\$k"
            cd "tool_output/\$d/\$m/k_\$k"
            echo -e "\nTool: \$m\nDataset: \$d\n k=\$k\n"
            if [ "\$m" = "meme" ]; then
                meme "../../../../\$d.txt" -dna -oc . -nostatus -time 14400
                -mod zoops -nmotifs 10 -minw \$k -maxw \$k -objfun classic -revcomp 
                -markov_order 0 
            elif [ "\$m" = "streme" ]; then
                streme --verbosity 1 --oc . --dna --totallength
                4000000 --time 14400 --minw \$k --maxw \$k 
                --thresh 0.05 --align center  --p "../../../../\$d.txt"
            fi
            cd ../../../..
        done
    done
done
\end{verbatim}

    To calculate the scores run toolscore.sh . The shell script file is-
    \begin{verbatim}
export PATH=$HOME/meme/bin:$HOME/meme/libexec/meme-:$PATH
methods=("meme" "streme")
datasets=("hm03" "yst04r" "yst08r")
for m in ${methods[@]}; do
    for d in ${datasets[@]}; do
        for k in {8..15}; do
            mkdir -p "tool_output/$d/$m/k_$k"
            cd "tool_output/$d/$m/k_$k"
            echo -e "\nTool: $m\nDataset: $d\n k=$k\n"
            if [ "$m" = "meme" ]; then
                python3 "../../../../meme_score.py" meme.txt
            elif [ "$m" = "streme" ]; then
                python3 "../../../../streme_score.py" sites.tsv
            fi
            cd ../../../..
        done
    done
done
\end{verbatim}
    
\chapter{Results}
    \section{Exp. configuration}
    The motif outputs from the two third-party tools MEME and STREME were parsed into motif matrices. The standard mismatch-scoring similar to Gibbs sampling and Randomized Motif Search was applied on the outputs of MEME and STREME to compare them with the methods we implemented.

\newpage
\section{Comparison}
    
\begin{table}[htbp]
\centering
\caption{Performance of Meme Tool on Various Datasets}
\begin{tabular}{@{}llccc@{}}
\hline
\textbf{Dataset} & \textbf{\(k\)} & \textbf{Score} & \textbf{Consensus} \\ \midrule
hm03 & 8 & 4 & TGGACCCA \\
hm03 & 9 & 7 & TCTGTCTCT \\
hm03 & 10 & 11 & CTCTGTCTCT \\
hm03 & 11 & 14 & GTCTCTGTCCC \\
hm03 & 12 & 17 & TGTCTCTGTCCC \\
hm03 & 13 & 18 & AATGAAAAAAAAA \\
hm03 & 14 & 26 & GCTCTGACACTCAC \\
hm03 & 15 & 26 & GAATGAAAAAAAAAT \\
yst04r & 8 & 2 & TTTCTGGC \\
yst04r & 9 & 4 & TTTTCTGGC \\
yst04r & 10 & 8 & CTTTTCTGGC \\
yst04r & 11 & 7 & TCTTTTCTTCC \\
yst04r & 12 & 4 & TTTTTTTTTCTT \\
yst04r & 13 & 10 & TTTTTTCTTCCTT \\
yst04r & 14 & 8 & TTTTTTTTCTTTTC \\
yst04r & 15 & 9 & TTTTTTTTTCTTTTC \\
yst08r & 8 & 5 & TTTCTGGC \\
yst08r & 9 & 4 & TTTCTTTTT \\
yst08r & 10 & 7 & GAAAAAAAAA \\
yst08r & 11 & 6 & TTTTTTTTTTT \\
yst08r & 12 & 9 & AAAAAAAAAAAT \\
yst08r & 13 & 16 & TTTTTTTTTTTCC \\
yst08r & 14 & 20 & AGGAAAAAAAAAAA \\
yst08r & 15 & 23 & TTTTTTTTTTTTTCC \\ \bottomrule
\hline
\end{tabular}
\end{table}

\begin{table}[htbp]
\centering
\caption{Performance of Streme Tool on Various Datasets}
\begin{tabular}{@{}llccc@{}}
\toprule
\textbf{Dataset} & \textbf{\(k\)} & \textbf{Score} & \textbf{Consensus} \\ \midrule
hm03 & 8 & 8 & AAACAAAG \\
hm03 & 9 & 10 & AGACACAGA \\
hm03 & 10 & 12 & AAAGAAGAAT \\
hm03 & 11 & 12 & GAGACTCACAG \\
hm03 & 12 & 22 & AAACAAAGYGAA \\
hm03 & 13 & 25 & AAAACAAAGCGAA \\
hm03 & 14 & 24 & GAATGAAAAMAAAA \\
hm03 & 15 & 32 & AAAAAAAAAGTGAAR \\
yst04r & 8 & 6 & AATTAATT \\
yst04r & 9 & 28 & WRATTAATYW \\
yst04r & 10 & 10 & AAGGTATATA \\
yst04r & 11 & 12 & ACAAGAGAGAA \\
yst04r & 12 & 15 & AAAAAATTAAKR \\
yst04r & 13 & 8 & AGAAAAGAAAAAA \\
yst04r & 14 & 18 & ARAAAARAAAAATA \\
yst04r & 15 & 12 & AGAAAAGAAAAAAAA \\
yst08r & 8 & 9 & GTAAATAA \\
yst08r & 9 & 8 & AAGAAAAGA \\
yst08r & 10 & 10 & AGGAAGAAAA \\
yst08r & 11 & 13 & GAAAAGAAAAA \\
yst08r & 12 & 8 & AAAAAAAAAAAT \\
yst08r & 13 & 11 & AAAAAAAAAAATA \\
yst08r & 14 & 18 & AAAAAAAAAAATAG \\
yst08r & 15 & 28 & AAAAAAAAAAATAGN \\ \bottomrule
\end{tabular}
\end{table}

\begin{table}[htbp]
\centering
\caption{Performance of Gibbs Sampler on various Datasets}
\label{tab:summary-results}
\begin{tabular}{@{}lccc@{}}
\toprule
\textbf{Dataset} & \textbf{\(k\)} & \textbf{Score} & \textbf{Consensus} \\ \midrule
hm03 & 8 & 9 & AAAAAAAA \\
hm03 & 9 & 12 & AAAATAAAA \\
hm03 & 10 & 15 & AAAAAAATAA \\
hm03 & 11 & 19 & AATGAAAAAAA \\
hm03 & 12 & 19 & AAAAAAAATAAA \\
hm03 & 13 & 24 & AAAAAAAATAAAA \\
hm03 & 14 & 26 & AATGAAAAAAAAAT \\
hm03 & 15 & 28 & AGAAAAAAAAATAAA \\
yst04r & 8 & 2 & TTTTTTTT \\
yst04r & 9 & 6 & AAAAAAAAA \\
yst04r & 10 & 6 & TTTTTTTTCT \\
yst04r & 11 & 8 & AAAAAAAAAAA \\
yst04r & 12 & 9 & AAAAAAAAAAAA \\
yst04r & 13 & 10 & TTATTTTTCTTTT \\
yst04r & 14 & 14 & AAAAAAAAAAAACA \\
yst04r & 15 & 14 & TTATTTTTCTTTTTT \\
yst08r & 8 & 6 & TTTTTTTT \\
yst08r & 9 & 8 & AAGAAAAAA \\
yst08r & 10 & 9 & TTTTTTTTTT \\
yst08r & 11 & 12 & ATTTTTTTTTT \\
yst08r & 12 & 13 & TATTTTTTTTTT \\
yst08r & 13 & 21 & AAAAAAAAAAAAA \\
yst08r & 14 & 27 & AAAAAAAAAAAAAA \\
yst08r & 15 & 30 & TTTTATTTTTTTTTT \\ 
\bottomrule
\end{tabular}
\end{table}

\begin{table}[htbp]
\centering
\caption{Performance of Modified Gibbs Sampler on various Datasets}
\label{tab:modified-gibbs-summary}
\begin{tabular}{@{}lccc@{}}
\toprule
\textbf{Dataset} & \textbf{\(k\)} & \textbf{Score} & \textbf{Consensus} \\ \midrule
hm03 & 8 & 10 & AAACAAAA \\
hm03 & 9 & 17 & GAGAAAACA \\
hm03 & 10 & 25 & ACTAAAACAC \\
hm03 & 11 & 28 & AGTTTTCTTTC \\
hm03 & 12 & 23 & AAGAAAAATCAA \\
hm03 & 13 & 29 & AAATGGAAAAGAG \\
hm03 & 14 & 38 & TAGATAAAAAAATA \\
hm03 & 15 & 45 & TCCCTTGAGCCCAGG \\
yst04r & 8 & 5 & TCTTTCTT \\
yst04r & 9 & 10 & TTTTTTTTC \\
yst04r & 10 & 16 & TTTGCATGTA \\
yst04r & 11 & 10 & TTCTTTTTTTT \\
yst04r & 12 & 14 & CATATATAAATA \\
yst04r & 13 & 18 & AAGGAAAAAAAAA \\
yst04r & 14 & 15 & TTTATTTTTTTTTT \\
yst04r & 15 & 24 & TTTTTTCTTAAAACT \\
yst08r & 8 & 14 & AAAAATAA \\
yst08r & 9 & 12 & TATTTTTTT \\
yst08r & 10 & 21 & AAAATTATTT \\
yst08r & 11 & 15 & TTTATTTTTCT \\
yst08r & 12 & 22 & TTTTTTTTTCTT \\
yst08r & 13 & 29 & CAAAAAAAAAAAA \\
yst08r & 14 & 37 & ATTTTTCTTCTCCA \\
yst08r & 15 & 37 & ATGAAAAAAGAAAAA \\
\bottomrule
\end{tabular}
\end{table}


\begin{table}[htbp]
\centering
\caption{Performance of Enhanced Randomized-Motif on various Datasets}
\label{tab:randomized-motif-summary}
\begin{tabular}{@{}lccc@{}}
\toprule
\textbf{Dataset} & \textbf{\(k\)} & \textbf{Score} & \textbf{Consensus} \\ \midrule
hm03 & 8 & 6 & CTCTGTCC \\
hm03 & 9 & 12 & TCTCCTTCC \\
hm03 & 10 & 18 & TGGAAGAGAG \\
hm03 & 11 & 19 & ATGAAAAAAAA \\
hm03 & 12 & 25 & ATGGAAAAGATA \\
hm03 & 13 & 27 & AGAAAGAGAGAAA \\
hm03 & 14 & 30 & AGAAAAAGAGAAAG \\
hm03 & 15 & 31 & AGCCAACAAAATAAA \\
yst04r & 8 & 1 & ATTTTTTT \\
yst04r & 9 & 5 & ATTTTTTTT \\
yst04r & 10 & 6 & TTTTTTTTCT \\
yst04r & 11 & 14 & ATCCTTTTCTT \\
yst04r & 12 & 12 & AAAAAAAAAAAA \\
yst04r & 13 & 14 & ATTTTTTTTCTTT \\
yst04r & 14 & 23 & AAGAAGAAAAAAAA \\
yst04r & 15 & 22 & AAAAAAAAAAAAAAA \\
yst08r & 8 & 4 & ATTTTTTT \\
yst08r & 9 & 6 & ATTTTTTTT \\
yst08r & 10 & 9 & ATTTTTTTTT \\
yst08r & 11 & 12 & TATTTTTTTTT \\
yst08r & 12 & 16 & ATTTTTTTTTTT \\
yst08r & 13 & 20 & CTATTTTTTTTTT \\
yst08r & 14 & 27 & ATTTTTTTTTTTTT \\
yst08r & 15 & 32 & AAAAAAAAAAAAAAA \\
\bottomrule
\end{tabular}
\end{table}


\begin{table}[htbp]
\centering
\caption{Performance of Randomized-Motif on hm03 Dataset}
\label{tab:motif-discovery-summary}
\begin{tabular}{@{}lcc@{}}
\toprule
\textbf{\(k\)} & \textbf{Score} & \textbf{Consensus} \\ \midrule
8 & 5 & AAAATAAA \\
9 & 9 & CTCTGTCCC \\
10 & 12 & GACACAGGGA \\
11 & 15 & GACACAGGGAG \\
12 & 18 & AAAAAAAATAAA \\
13 & 21 & AAAAAAATAAAAA \\
14 & 24 & AGCAAACAAAATAA \\
15 & 28 & AGCAAACAAAATAAA \\
\bottomrule
\end{tabular}
\end{table}


\begin{table}[htbp]
\centering
\caption{Summary of Gibbs Sampler Exempt Results}
\label{tab:gibbs-exempt-summary}
\begin{tabular}{@{}lccc@{}}
\toprule
\textbf{Dataset} & \textbf{\(k\)} & \textbf{Score} & \textbf{Consensus} \\ \midrule
hm03 & 8 & 7 & AAAATAAA \\
hm03 & 9 & 12 & AAAAAAAAA \\
hm03 & 10 & 13 & AAAATAAAAA \\
hm03 & 11 & 16 & AAAAAAAATAA \\
hm03 & 12 & 17 & AAAAAAAATAAA \\
hm03 & 13 & 21 & AAAAAAAATAAAA \\
hm03 & 14 & 24 & AATGAAAAAAAAAT \\
hm03 & 15 & 27 & AGTAAACAAAATAAA \\
yst04r & 8 & 3 & TTTTTTTT \\
yst04r & 9 & 4 & ATTTTTTTT \\
yst04r & 10 & 5 & TTTTTTTTCT \\
yst04r & 11 & 7 & TATTTTTCTTT \\
yst04r & 12 & 8 & TATTTTTCTTTT \\
yst04r & 13 & 9 & TTATTTTTCTTTT \\
yst04r & 14 & 14 & AAAAAAAAAAAACA \\
yst04r & 15 & 14 & TTATTTTTCTTTTTT \\
yst04r &9 & 8 & AAAAAAAAA \\
10 & 9 & TTTTTTTTTT \\
11 & 13 & AAAAAAAAAAA \\
12 & 13 & TATTTTTTTTTT \\
13 & 17 & TATTTTTTTTTTT \\
14 & 23 & TATTTTTTTTTTTT \\
15 & 30 & AAAAAAAAAAAAAAA \\
\bottomrule
\end{tabular}
\end{table}
    
\chapter{Conclusion}


Throughout this work, we have done a comprehensive exploration of motif discovery algorithms, with a particular focus on the Gibbs Sampler and Randomized Motif Search methodologies. Our journey included the implementation of standard approaches, innovative modifications to enhance their effectiveness, and the application of third-party tools like MEME and STREME for comparative analysis.

\section{Insights Gained}

The modifications to the Gibbs Sampler algorithm, designed to introduce a more targeted selection of sequences for updating and the incorporation of exempt sequences, have shown promising results. 
Similarly, the Enhanced Randomized Motif Search strategy, which emphasizes multiple iterations from varied initial conditions, highlighted the importance of exploring the motif space comprehensively. By avoiding premature convergence on suboptimal motifs, this approach underlines the stochastic nature of biological data analysis.

\section{Challenges Encountered}

One of the main challenges in motif discovery remains the validation of identified motifs due to the absence of a universally accepted "ground truth." The reliance on tools like Tomtom for comparison against known motifs databases underscores the difficulty in ensuring the biological relevance of newly discovered motifs. 

\section{Future Directions}

Looking ahead, there are several paths for further research and development in the field of motif discovery. The integration of machine learning techniques, particularly deep learning, offers an exciting road for the development of algorithms that can learn complex patterns and predict motifs with higher accuracy. 




\renewcommand{\bibname}{References}
\bibliographystyle{apalike}
\bibliography{refs} % Specify your .bib file here

\end{document}
